%recursion_tail.tex 
%auxilary notes for the course algorithms COMS10007
%taught at the University of Bristol 2018 Conor Houghton
%conor.houghton@bristol.ac.uk

%this is material that appeared in a previous version of the course

%To the extent possible under law, the author has dedicated all copyright 
%and related and neighboring rights to these notes to the public domain 
%worldwide. These notes are distributed without any warranty. 

\documentclass[11pt,a4paper]{scrartcl}
\typearea{12}
\usepackage{graphicx}
\usepackage{listings}
\lstset{language=C}
\pagestyle{headings}
\markright{COMS10007 - algorithms recursion\_tail - Conor}
\begin{document}



\subsubsection*{Recursion: tail recursion}
     
A previous version of this course was taught in the first rather than
the second term and so it was useful to deal with tail recursion. This
material has now been removed, but can be see here.

The factorial also provides a convenient example for discussing tail
recursion. One problem with recursion is that it can be wasteful of
stack memory, roughly speaking the memory the program uses to run. If
you call \texttt{factorial(10)} the program will write information
related to \texttt{factorial(10)}, \texttt{factorial(9)} and so on
down to \texttt{factorial(1)} onto the stack before
\texttt{factorial(1)} returns and the open functions get rolled, being
deleted from memory. In short, a copy of \texttt{factorial(9)} needs
to be stored while it waits for \texttt{factorial(8)} to return the
value 40320 to it, so that it can multiply that by 9 and return the
result, 362880, on to \texttt{factorial(10)}. This isn't really a
problem here, the factorial function will reach values well beyond the
capacity of int long long before the memory use on the stack becomes a
problem; however, in other circumstances it might be a problem. The
solution is tail recursion.


\begin{table}[b]
\begin{lstlisting}[numbers=left]
int factorial(int n)
{
   if(n<2)
      return 1;

   return n*factorial(n-1);
}

\end{lstlisting}
\caption{The recursive function for calculating $n!=n(n-1)\ldots 1$. If $n<2$ it returns 1, giving a terminating condition, it also means $0!=1$ which is a normal mathematical convention, otherwise it calls factorial(n-1). If you trying using this function, note that for even modest values of n, n! is too big to fit into int.\label{c_factorial}}
\end{table}


\begin{table}
\begin{lstlisting}[numbers=left]
int factorial(int n)
{
   return (n<2) ? 1 : n*factorial(n-1);
}

\end{lstlisting}
\caption{A fancier version of the factorial program which uses the ternary operator described in Table~\ref{c_ternary}.\label{c_factorial_fancy}}
\end{table}


\begin{table}
\begin{lstlisting}[numbers=left]
if (a)
   ans=b;
else 
   ans=c;
\end{lstlisting}
\caption{The ternary operator ans = a ? b : c evaluates a and then
  does either sets ans=b or ans=c depending on whether a is true of
  false.  Thus ans=a ? b : c is equivalent to the code above. Ternary
  operators are often faster to execute than the corresponding if
  statement.\label{c_ternary}}
\end{table}
 
An algorithm is tail recursive if the return value of the function
does not have to be modified before it is returned. The factorial
function in Table~\ref{c_factorial} is not tail recursive because
factorial(n-1) is multiplied by n before it is returned. However, the
version in Table~\ref{c_factorial_tail} is tail recursive, it manages
this by passing around another variable called big\_n that holds the
part of the factorial that is calculated so far. Now, when
\texttt{factorial(10)}, for example, is called it will call
\texttt{factorial(10,1)}, since $10>2$ this will call
\texttt{factorial(9,10)}; the only thing remaining for
\texttt{factorial(10,1)} to do is to return the value of
\texttt{factorial(9,10)} when it is done. This means, with a bit of
cunning, the function does not have to remain written on the stack,
the compiler just has to know that whatever \texttt{factorial(9,10)}
returns should be sent to whatever called \texttt{factorial(10)};
modern compilers are capable of this cunning so tail recursive
algorithms make more efficient use of stack memory.

\begin{table}
\begin{lstlisting}[numbers=left]
int factorial_r(int n, int big_n)
{
  if(n<2)
    return big_n;

  return factorial_r(n-1,n*big_n);
}

int factorial(int n)
{
  return factorial_r(n,1);
}
\end{lstlisting}
\caption{The tail recursive function for calculating $n!=n(n-1)\ldots
  1$. If $n<2$ it returns big\_n, otherwise it calls
  factorial(n-1,n*big\_n). Since nothing happens to the
  factorial(n-1,n*big\_n) before it is itself returned, this is an
  example of tail recursion.\label{c_factorial_tail}}
\end{table}

\begin{table}
\texttt{factorial\_r(10,1)}$\rightarrow$\texttt{factorial\_r(9,10)}$\rightarrow$\texttt{factorial\_r(8,90)}$\rightarrow$\\
\texttt{factorial\_r(7,720)}$\rightarrow$\texttt{factorial\_r(6,5040)}$\rightarrow$\texttt{factorial\_r(5,30240)}$\rightarrow$\\
\texttt{factorial\_r(4,151200)}$\rightarrow$\texttt{factorial\_r(3,604800)}$\rightarrow$\texttt{factorial\_r(2,1814400)}$\rightarrow$\\
\texttt{factorial\_r(1,3628800)}
\caption{The calling sequence of the tail recursive factorial program.\label{calling}}
\end{table}

\end{document}
