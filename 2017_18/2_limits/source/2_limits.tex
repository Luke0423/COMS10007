%2_limits.tex
%notes for the course Algorithms COMS10007 taught at the University of Bristol
%2017_18 Conor Houghton conor.houghton@bristol.ac.uk

%To the extent possible under law, the author has dedicated all copyright 
%and related and neighboring rights to these notes to the public domain 
%worldwide. These notes are distributed without any warranty. 

\documentclass[11pt,a4paper]{scrartcl}
\typearea{12}
\usepackage{graphicx}
%\usepackage{pstricks}
\usepackage{listings}
\usepackage{color}
\lstset{language=C}
\pagestyle{headings}
\markright{COMS10007 1\_limits (d) - Conor}
\begin{document}

\section*{2: Limits}

Limits are the mathematical framework for deciding where a function is
going as it argument approaches some value. In the case of algorithmic
complexity, we are interested in where $T(n)$, the run time, is going,
as $n$ gets large so here we will look at limits at infinity. The
limit is a part of mathematics where the intuition is probably more
straight forward than the definition, but here we will look at the
formal definition. This is useful not so much for this course, largely
we will be able to calculate the limits we need using simple methods,
but because it is something that is useful to know in the future.

\subsection*{Informal idea}

If we write
\begin{equation}
\lim_{x\rightarrow \infty}f(x)=c
\end{equation}
where $f(x)$ is some function and $c$ a constant, we mean that $f(x)$
heads towards infinity it gets close to $c$. Take, for example
\begin{equation}
f(x) = \frac{1}{x}
\end{equation}
it is easy to see that 
\begin{equation}
\lim_{x\rightarrow \infty}\frac{1}{x}=0
\end{equation}
because the bigger $x$ is, the smaller $1/x$ is, so $1/x$ is heading for zero. Now, lets consider 
\begin{equation}
f(x)=\frac{4x^2+2x+1}{2x^2+3}
\end{equation}
Well by dividing above and below by $x^2$ we have
\begin{equation}
\lim_{x\rightarrow \infty}\frac{4x^2+2x+1}{2x^2+3}=
\lim_{x\rightarrow \infty}\frac{4+2/x+1/x^2}{2+3/x^2}=
\lim_{x\rightarrow \infty}\frac{4}{2}=2
\end{equation}
because the $2/x$, $3/x^2$ and so on are getting smaller and smaller
as $x$ gets larger so we can replace them by zeros inside the limit.

Of course, many functions just get bigger and bigger, or more and more
negative, as $x$ goes to infinity, for these we say the limit is plus
or minus infinity, so, for example
\begin{equation}
\lim_{x\rightarrow \infty} x=\infty
\end{equation}
and
\begin{equation}
\lim_{x\rightarrow \infty} (-x)=-\infty
\end{equation}



\end{document}

